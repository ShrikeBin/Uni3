\documentclass{article}
\usepackage{amsmath}
\usepackage{amssymb}
\usepackage{graphicx}
\usepackage{listings}
\usepackage{xcolor}
\usepackage{geometry}
\geometry{a4paper, margin=1in}

\title{Raport z Symulacji Monte Carlo}
\author{}
\date{}

\begin{document}

\maketitle

\section*{Opis wyników}

W ramach zadania wykorzystałem metodę Monte Carlo do przybliżonego obliczenia wartości całek oznaczonych dla następujących funkcji:
\begin{itemize}
    \item \( f(x) = \sin(x) \) na przedziale \( [0, \pi] \),
    \item \( g(x) = 4x(1 - x)^3 \) na przedziale \( [0, 1] \),
    \item \( h(x) = \sqrt[3]{x} \) na przedziale \( [0, 8] \),
    \item \( \pi(x, y) = \begin{cases} 1 & \text{} x^2 + y^2 < 1 \\ 0 & \text{else} \end{cases} \) w kwadracie jednostkowym (przybliżanie wartości \(\pi\)).
\end{itemize}

Na wykresach (pliki .png) przedstawiam wyniki pojedynczych symulacji oraz ich średnie wartości, umożliwiając porównanie przybliżonych wyników Monte Carlo z wartościami faktycznymi.

\section*{Wnioski}

\begin{enumerate}
    \item \textbf{Dokładność przybliżeń}: \\
    Metoda Monte Carlo skutecznie przybliża wartości całek dla wszystkich analizowanych funkcji. Wraz ze wzrostem liczby próbek \( n \) rośnie stabilność wyników, co pozwala uzyskać przybliżenia bliższe wartości rzeczywistej.
    
    \item \textbf{Wpływ liczby powtórzeń}: \\
    Zwiększenie liczby powtórzeń poprawia stabilność wyników, zmniejszając odchylenia od wartości rzeczywistej. Widać przez to, że liczba powtórzeń jest kluczowa dla dokładności wyników.
    
    \item \textbf{Charakterystyka zwiększania ilości powtórzeń dla \( n\)}: \\
    Warto zauważyć że dla dużych wartości \( n\) wynik zaczylają oscylować wokół wartości faktycznej w stałym zakresie co obrazuje że zwiększanie liczby powtórzeń nie daje lepszego przybliżenia oraz że lepszym sposobem jest branie średniej z tych wyników.
    
    \item \textbf{Efektywność obliczeniowa}: \\
    Choć metoda Monte Carlo jest prosta w implementacji, wymaga dużej mocy obliczeniowej dlatego powinna być używana by w relatywnie dokładny sposób przybliżać wartości o średniej dokładności.
\end{enumerate}

\section*{Podsumowanie}

Symulacja Monte Carlo pozwala uzyskać wartości bliskie rzeczywistym dla wybranych funkcji, stanowiąc skuteczne narzędzie w przypadkach, gdy analityczne obliczenia są trudne. Zwiększenie liczby próbek i powtórzeń znacząco poprawia dokładność wyników, potwierdzając efektywność tej metody.
\vspace{1cm} 

\hfill
\textbf{Jan Ryszkiewicz}

\end{document}
