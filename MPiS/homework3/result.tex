\documentclass{article}
\usepackage{graphicx}
\usepackage{amsmath}
\usepackage{pgfplots}
\usepackage{tikz}

\title{Sprawozdanie z realizacji zadań 1-3}
\author{Jan Ryszkiewicz}
\date{\today}

\begin{document}

\maketitle

\section*{Zadanie 1: The Power of Two Choices / balanced allocation}

Testowałem dwa przypadki:

\begin{itemize}
    \item (a) Dla każdej kuli wybieramy niezależnie i jednostajnie losowo jedną z \(n\) urn, w której umieszczamy kulę.
    \item (b) Dla każdej kuli wybieramy niezależnie i jednostajnie losowo dwie urny, a kulę umieszczamy w najmniej zapełnionej z wybranych urn.
\end{itemize}

Po przeprowadzeniu symulacji dla \(n \in \{10 000, 20 000, \dots, 1 000 000\}\) \\
oraz obliczeniu średniego maksymalnego zapełnienia \(L^{(d)}_n\) dla obu przypadków (gdzie \(d = 1\) i \(d = 2\)), \\
uzyskano wyniki, które zaprezentowano na wykresach (exc1/plots/*). \\
Dodatkowo, wykresy przedstawiają funkcje \L(n) podzieloną przez:
\begin{itemize}
    \item \(f_1(n) = \frac{\ln n}{\ln \ln n}\)
    \item \(f_2(n) = \frac{\ln \ln n}{\ln 2}\)
\end{itemize}

Na podstawie wykresów stwierdzono, że rozkład wyników dla przypadku drugiego jest skoncentrowany wokół wartości średniej, \\
jednak w przypadku \(d = 1\) rozkład jest bardziej chaotyczny, co sugeruje większe zróżnicowanie wyników. \\
Asymptotycznie wartości \(L^{(d)}_n\) przybliżają się do wartości \(\frac{\ln n}{\ln \ln n}\) dla \(d = 1\) oraz \(\frac{\ln \ln n}{\ln 2}\) dla \(d = 2\).\\

\section*{Zadanie 2: Sortowanie przez wstawianie losowych danych}

W zadaniu 2 zaimplementowaliśmy algorytm sortowania przez wstawianie (INSERTIONSORT) i przeprowadziliśmy \\
eksperymenty dla \(n \in \{100, 200, \dots, 10 000\}\) oraz \(k = 50\) powtórzeń dla każdego \(n\). \\
Dla każdego eksperymentu zebrano dane dotyczące liczby wykonanych porównań oraz przestawień kluczy.\\

Na podstawie zebranych danych przedstawiono wykresy (exc2/plots/*). \\

Z wykresów wynika, że zarównoliczba porównań jak i przestawień rośnie niemal kwadratowo w zależności od \(n\), \\
Wartości ilorazów \(\frac{cmp(n)}{n}\) oraz \(\frac{cmp(n)}{n^2}\) \\
wskazują na kwadratową złożoność czasową algorytmu w przypadku sortowania przez wstawianie.\\

\section*{Zadanie 3: Uproszczony model komunikacji z zakłóceniami}

W zadaniu 3 przeprowadzono symulacje w celu eksperymentalnego zbadania minimalnej liczby rund \(T_n\) potrzebnej do rozesłania informacji w sieci 
o topologii gwiazdy z zakłóceniami. Eksperymenty przeprowadzono dla \(p = 0.5\) oraz \(p = 0.1\).\\

Na podstawie uzyskanych wyników przedstawiono wykresy liczby rund potrzebnych do rozesłania informacji (exc3/plots/*).\\

Z wykresów wynika, że liczba rund potrzebnych do rozesłania informacji rośnie wraz z \(n\), ale dla mniejszego prawdopodobieństwa \(p\) liczba rund rośnie szybciej. \\
Potwierdza to, że mniejsze prawdopodobieństwo odbioru informacji wydłuża czas potrzebny do jej rozesłania w sieci.\\

\end{document}
